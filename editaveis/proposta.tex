%\part{Aspectos Gerais}

\chapter[Proposta Inicial]{Proposta Inicial}

\section{Contextualização}
	A qualidade na produção de software é uma área muito ampla e que abrange desde qualidade da arquitetura de software até qualidade no processo. Este trabalho teve como objetivo central apresentar um estudo sobre uma arquitetura de qualidade dentro de alguns orgãos, tendo como os principais pilares três áreas comuns da engenharia de software, gerência de configuração, integração contínua e análise estática de código. Uma arquitetura próxima a essa vem sendo trabalhada dentro de alguns órgãos públicos do governo federal, entre eles o Tribunal de Contas da União o qual tem mostrado os melhores resultados nesta área. O problema encontrado atualmente está na falta de um acompanhamento na qualidade dos chamados softwares legados, softwares que são produzidos dentro/para o órgão e que passado um tempo ainda estão em atividade.
	Dentro da área de Tecnologia as organizações procuram terceirizar determinados serviços com o objetivo de 
	
	
\section{Problema de Pesquisa}
	O principal produto da engenharia de software é o software, contudo o que tem se vivenciado na realidade brasileira de computação é que o software que está sendo entregue é um software precário e de baixa qualidade. Por ser uma palavra abstrata, o conceito de qualidade é bem amplo, porém o termo qualidade normalmente está associado a uma medida relativa, essa qualidade pode ser entendida como “conformidade às especificações”. Conceituando dessa forma, a não conformidade às especificação é igual a ausência de qualidade [1].
	
	
	

	
\section{Justificativa}

	

	\section{Objetivos}

	\subsection{Objetivos Gerais} % (fold)
	\label{sub:objetivos_gerais}
	
		Adaptar técnicas de resolução do problema de SLAM para o contexto da Robótica Educacional, utilizando os kits de robótica \textit{Mindstorms} da Lego.

	% subsection objetivos_gerais (end)

	\subsection{Objetivos Específicos} % (fold)
	\label{sub:objetivos_específicos}

		O Objetivo Geral pode ser dividido em quatro objetivos específicos, que vão desde a solução do problema até a adaptação da mesma pro contexto educacional, como são apresentados a seguir.
		 
	\begin{itemize}
		\item Resolução do problema de SLAM;
		\item Adaptação pro contexto educacional;
		\item Garantia da qualidade da solução (Visão da Engenharia de Software); 
	\end{itemize}
	
	% subsection objetivos_específicos (end)