%\part{Aspectos Gerais}

\chapter[Proposta Inicial]{Proposta Inicial}
	


\section{Contextualização}
	A qualidade na produção de software é uma área muito ampla e que abrange desde qualidade da arquitetura de software até qualidade no processo. Este trabalho teve como objetivo central apresentar um estudo sobre uma arquitetura de qualidade dentro de alguns orgãos, tendo como os principais pilares três áreas comuns da engenharia de software, gerência de configuração, integração contínua e análise estática de código. Uma arquitetura próxima a essa vem sendo trabalhada dentro de alguns órgãos públicos do governo federal, entre eles o Tribunal de Contas da União o qual tem mostrado os melhores resultados nesta área. O problema encontrado atualmente está na falta de um acompanhamento na qualidade dos chamados softwares legados, softwares que são produzidos dentro/para o órgão e que passado um tempo ainda estão em atividade.
	\\Dentro da área de Tecnologia as organizações tem encontrado um grande problema quando se trata de softwares legados. O trabalho feito por Croazara (2014) prova que os softwares legados são esquecidos pelas organizações quando se fala sob uma perspectiva de manutenção. Contudo o estudo também revela que em grande parte das empresas esses mesmos softwares continuam rodando no ambiente de produção.
	
	
	
\section{Problema de Pesquisa}
	O principal produto da engenharia de software é o software, contudo o que tem se vivenciado na realidade brasileira de computação é que o software que está sendo entregue é um software precário e de baixa qualidade. Por ser uma palavra abstrata, o conceito de qualidade é bem amplo, porém o termo qualidade normalmente está associado a uma medida relativa, essa qualidade pode ser entendida como “conformidade às especificações”. Conceituando dessa forma, a não conformidade às especificação é igual a ausência de qualidade.
	\\Uma das grandes dificuldades nos orgãos públicos está no acompanhamento das manutenções prestadas por terceirizadas. Esse problema se agrava ainda mais quando a empresa contratante não consegue acompanhar ou não tem parametros concretos de indicadores de qualidade. Este trabalho tem como proposta a criação de uma dashboard de monitoramento para softwares legados onde é possível acompanhar de maneira simples e totalmente visual, indicadores de qualidade de código para projetos selecionados.
	\\O grande desafio está em criar uma maneira de visualização de dados de maneira simplificada e objetiva para acompanhamento de software em um orgão público federal. Tomando este problema como base tem-se a seguinte questão de pesquisa:
	
	\begin{center}
	\textit{"Uma vez definido os indicadores das métricas, como criar uma interface de visualização da informação para um orgão público federal? }"	
	\end{center}
	
\section{Justificativa}
	A motivação deste trabalho se deu como uma extensão de um trabalho de conclusão de concurso elaborado anteriormente em que eram tratados aspectos para monitoramento da qualidade dentro de um orgão público federal. O trabalho abordava aspectos de contratação de software dentro desses orgãos e como acontecia o acompanhamento desses softwares e propunha uma solução com integração contínua e gerência de configuração.
	\\Após a conclusão do trabalho citado algumas lacunas continuaram, essas lacunas estão ligadas à apresentação destes dados para a equipe de gestão com o intuito de facilitar o acompanhamento das métricas.


	

	\section{Objetivos}

	\subsection{Objetivos Gerais} % (fold)
	\label{sub:objetivos_gerais}
	
		Criação de solução intuitiva para monitoramento da qualidade de código de softwares legados atendendo as necessidades de um orgão X

	% subsection objetivos_gerais (end)

	\subsection{Objetivos Específicos} % (fold)
	\label{sub:objetivos_específicos}

	Para que seja possivel alcancar o objetivo geral alguns outros objetivos menores precisam ser alcancados para garantir  o objetivo geral 
		 
	\begin{itemize}
		\item Identificar metricas de codigo ja existentes que mais se adequem as necessidades do Orgao X
		\item Propor um ambiente integralizado e automatizado, englobando solucoes de analise estatica de codigo, integracao continua e versionamento de codigo
		\item Uma solucao que agregue valor ao orgao X
	\end{itemize}
	
	% subsection objetivos_específicos (end)